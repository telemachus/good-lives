% [[- LaTeX prelude
\documentclass[12pt,letterpaper]{article}

\usepackage[no-math]{fontspec}
\setmainfont{Baskerville}

% Ugh: https://tex.stackexchange.com/a/400994/29387
% This no longer seems necessary, but I’ll leave it for a bit.
% \usepackage[base]{babel}
\usepackage[nolocalmarks]{polyglossia}
\setdefaultlanguage{english}
\setotherlanguage[variant=classic]{latin}
\setotherlanguage[variant=ancient]{greek}
\newfontfamily\greekfont[Script=Greek,Scale=MatchLowercase]{GFS Neohellenic}

\usepackage{titlesec}
\titleformat*{\section}{\large\bfseries}
\titleformat*{\subsection}{\bfseries}
\titleformat*{\subsubsection}{\bfseries}

\usepackage{parskip}
\usepackage{csquotes}
\usepackage[style=windycity,citetracker=context,backend=biber]{biblatex}
\addbibresource{good-lives.bib}

\usepackage{enumitem}
\setlist{noitemsep}
\usepackage[super]{nth}

\begin{hyphenrules}{latin}
    \hyphenation{}
\end{hyphenrules}

\begin{hyphenrules}{greek}
    \hyphenation{}
\end{hyphenrules}

\usepackage{fancyhdr}
\fancypagestyle{notes}{%
    \fancyhf{}
    \renewcommand{\headrulewidth}{0pt}
    \lhead{}
    \chead{Notes on ``Pleasure and desire''}
    \rhead{}
    \lfoot{}
    \cfoot{\thepage}
    \rfoot{}
}
\fancypagestyle{references}{%
    \fancyhf{}
    \renewcommand{\headrulewidth}{0pt}
    \lhead{}
    \chead{Bibliography}
    \rhead{}
    \lfoot{}
    \cfoot{\thepage}
    \rfoot{}
}

% -]] Latex prelude

% [[- LaTeX document
\begin{document}

\raggedright

% [[- Title page
% \begin{titlepage}
% \title{Notes on Pleasure and desire (Raphael Woolf)}
% \author{Peter Aronoff}
% \maketitle
% \thispagestyle{empty}
% \end{titlepage}
% -]]

\pagestyle{notes}

% [[- Introduction
\section*{Introduction}

In this brief introduction, Woolf sets the stage for his argument.
He views Epicurus as a psychological hedonist (\textit{contra} Cooper) who defines pleasure as ``lack of pain in body (\textit{aponia}) and lack of distress in soul (\textit{ataraxia}).''%
\footcite[][158]{pleasure-and-desire-woolf-2009}
However, Woolf does not think that Epicurus rejected active, sensual pleasures in favor of an ascetic life.
% -]] Introduction

% [[- An Epicurean Mean? (Section I)
\section*{An Epicurean Mean? (Section I)}

Woolf contends that Epicurus argues for a mean between two bad extremes.
On the one hand, a true ascetic regards luxury as a positive evil, and as such avoids it at all costs.
On the other hand, someone who values luxury too much anxiously desires it and chafes when they cannot achieve it.
The wise Epicurus, on the other hand, takes luxuries if they are reasonably easy to get but does not miss them when they are not present or too hard to get.

To understand the bad extremes, Woolf uses two examples.
First, he refers to a story about Wittgenstein.
According to legend, Wittgenstein gave away his inherited wealth because he thought that wealth corrupted people.
However, because Wittgenstein didn't want his money to corrupt poor people, he gave the money to his own siblings.
Since his siblings were already wealthy, Wittgenstein reasoned that more money wouldn't hurt them as much as it would hurt poor people.
Second, Woolf invents a scenario about plane tickets.
He imagines a flight where there are some better tickets available for free.
Someone like Wittgenstein would reject the free tickets to avoid corruption.
Others might want the free tickets so much that they go to emotional extremes.
But Epicurus gets things just right, according to Woolf.
If Epicurus wins a free ticket, he enjoys it, but he doesn't worry much either way.
% -]] An Epicurean Mean? (Section I)

% [[- Epicurus and Luxury (Section II)
\section*{Epicurus and Luxury (Section II)}

In this section, Woolf makes his case concerning Epicurus and luxury.
Woolf offers \textit{Ep. Men.} 130 as his textual evidence.

\begin{quote}
    We consider self-sufficiency a great good, not in order that in all circumstances we use little, but so that, if we do not have much, we be satisfied with little, having been genuinely persuaded that luxury (\textit{poluteleia}) is most pleasant enjoyed by those who need it least, and that what is natural is all easy to procure, and what is empty is hard to procure.
\end{quote}

First, Woolf points out that the passage does not support an ascetic reading of Epicurus.
Epicurus clearly does not say that we should always avoid luxury or that luxury is bad \textit{per se}.
In addition, however, Woolf claims that Epicurus urges us to choose luxury when we can.
According to a common interpretation, Epicurus minimizes luxury even if he doesn't completely reject it.
Woolf, however, thinks that Epicurus makes room for as much luxury as one can get.
Woolf specifically rejects a claim by Brunschwig and Sedley that Epicurus recommended a life ``of simple frugality, punctuated with just occasional feasts and other indulgences.''%
\footcite[][161]{hellenistic-philosophy-brunschwig-sedley-2003}
Instead, Woolf argues that Epicurus believes (i) that luxuries will rarely come our way, but (ii) that we should enjoy them whenever we can, all the while (iii) understanding that luxuries are lovely but not necessary.
Thus, a good Epicurean is prepared for---and happy with---a simple life, but neither an ascetic or even someone who deliberately limits luxuries.

I am not going to argue with Woolf over interpretation.
First, Woolf does not engage much of the text.
Thus, there's no point in detailed argument about the text.
Second, Woolf suggests an interesting philosophical position.
Thus, regardless of whether Epicurus held Woolf's view, I think we should consider the view itself.

Let's evaluate the view then.
Woolf says that we should enjoy luxury whenever we can, but we should always remember that luxury is not necessary and that we can be happy without it.
At the limit, Woolf even imagines a person who enjoys luxury all the time.
Such a person does not need to enforce simplicity on themselves in order to get used to it.
Instead, people can use a ``purely psychological procedure'': they can ``simply repeat to oneself until fully internalized that if luxury were to be lost, one would be quite content with simplicity.''%
\footcite[][164]{pleasure-and-desire-woolf-2009}

I have several problems with Woolf's view.
First, according to Woolf, Epicurus insists that opportunities for luxury are rare.
As a result, Woolf does not worry that Epicureans may become dependent on luxuries.
Even if we assume that Epicurus was right about the ancient world, we need to reconsider his views now.
For example, consider cheap fried and heavily processed foods.
(And I suspect there are many other such cases.)
Second, I think Woolf is wrong: a person can't live a life of constant (or near constant) luxury and yet remain capable of simplicity simply by force of will and repetition.
I'm not saying that willpower and repetition are powerless, but Woolf makes things far too easy for himself.
% -]] Epicurus and Luxury (Section II)

% [[- Epicurus versus Luxury? (Section III)
\section*{Epicurus versus Luxury? (Section III)}

In this section, Woolf considers an objection: if Epicurus does not hate luxury, why does he sometimes rail against extreme pleasures.
Woolf blunts this criticism in two ways.
First, according to Woolf, Epicurus does not attack pleasures (as such), but people who (wrongly) believe that extreme pleasures are the goal of life.
(Woolf agrees that Epicurus describes the goal in negative terms as lack of mental and physical pain.)
Second, Epicurus criticizes people who pursue such pleasures without thinking of consequences.
According to Woolf, Epicurus does not speak against the pleasures themselves.
Here is a representative sample: ``we have a critique of a thoughtless approach rather than a type of pleasure (`luxurious') as such.''%
\footcite[][165]{hellenistic-philosophy-brunschwig-sedley-2003}
% -]] Epicurus versus Luxury? (Section III)

% [[- Epicurus, Luxury, and Empty Desires (Section IV)
\section*{Epicurus, Luxury, and Empty Desires (Section IV)}

Epicurus criticizes some desires as \textit{empty}, but would he classify all desires for luxury as empty?
Woolf replies that Epicurus does not and need not consider the desire for luxury to empty.
According to Woolf, Epicurus criticizes desires as empty only when they are intense and involve false beliefs (e.g., the belief that these luxury objects are necessary or very important for one's happiness).
However, says Woolf, Epicurus considers desires for, e.g., luxurious foods to be natural though non-necessary.

At this point, Woolf has to dance quickly and carefully.
He claims that a happy life plus luxury to be preferable and more pleasurable than a happy life without luxury.
However, Woolf denies that a happy life plus luxury is happier than a happy life without luxury.
Luxury does not increase happiness, but it does increase pleasure.
We choose luxury in such cases not because it makes our life better but because it is more pleasant than simpler goods.
But Woolf knows that he cannot attribute this view to Epicurus since Epicurus explicitly denies that pleasure can be increased past the point of painlessness.
% -]] Epicurus, Luxury, and Empty Desires (Section IV)

% [[- Epicurus and the Limit of Pleasure (Section V)
\section*{Epicurus and the Limit of Pleasure (Section V)}

In Woolf's view, Epicurus struggles with tensions in his ethics and offers two different solutions to these tensions.
According to Woolf, ``[t]he heart of the problem for a hedonist is that pleasure does appear to behave differently from happiness, and Epicurus seems to recognize this.''%
\footcite[][168]{hellenistic-philosophy-brunschwig-sedley-2003}
Pleasure and happiness differ insofar as pleasure seems able to increase even after happiness no longer does.
To return to Woolf's favorite example, if an airline passenger gets an upgrade, they get more pleasure, but we don't want to say that they get a happier life.
But, again according to Woolf, ``a hedonist ought to allow no measure of the quality of life other than pleasure.''%
\footcite[][168]{hellenistic-philosophy-brunschwig-sedley-2003}

This is where Woolf says that Epicurus takes two different approaches.

\begin{enumerate}
    \item Sometimes Epicurus bites the bullet: he simply denies that pleasure increases past the point of happiness.
    Thus, \textit{KD} 18 says that pleasure does not increase once pain is gone.
    This fixes things since pleasure and happiness are no longer different, but Woolf views this as an extreme measure since it's so implausible to deny that pleasure increases.
    (Note that Woolf does not think it's implausible to deny that happiness increases.)
    \item At other times, Epicurus solves his problem by distinguishing between two types of pleasure, one of which increases while the other does not.
    Kinetic pleasure (or pleasure in motion) increases without limit.
    Katastematic pleasure (or static pleasure) does not increase once all pain is gone.
    Epicurus can then identify happiness with static pleasure.
    We still have reason to seek kinetic pleasure, pleasure is still pleasant, but we should not care too much about it or misunderstand the role of such pleasures in the happy life.
\end{enumerate}
% -]] Epicurus and the Limit of Pleasure (Section V)

% [[- Why is Epicurus a Hedonist? (Section VI)
\section*{Why is Epicurus a Hedonist? (Section VI)}

Woolf rejects the argument that Epicurus should not be a hedonist; he also rejects the complaint that absence of pain and anxiety is itself a pleasure.
In a nutshell, Woolf argues (i) that Epicurus maintains a coherent position, (ii) that hedonism supports the cradle argument, and (iii) that the absence of pain \textit{is} a genuine pleasure.

To support (iii), Woolf notes that the problem may be (partly?) terminological and specific to ancient Greek.
Epicurus uses two terms, \textit{aponia} (\textgreek{ἀπονία}) and \textit{ataraxia} (\textgreek{ἀταραξία}), that have an alpha-privative in Greek.
This may suggest to an ancient Greek speaker or reader that these states are purely negative, without any characteristic sensory quality.
Woolf cites two ancient comments to this effect.
First, in Plato's \textit{Gorgias}, Socrates asks Callicles whether people who need nothing are happy, and Callicles replies that if they were, then ``rocks and corpses would be happiest'' (492e3--6).
Second, the Cyrenaics joked that Epicurean happiness was the happiness of a corpse (Clement of Alexandria, \textit{Stromata} 2.130) or a sleeping person (Diogenes Laertius, 2.89).

Nevertheless, Woolf maintains that Epicurus is correct.
He writes, ``What [Epicurus] is describing is not a neutral state, but one with a felt character that is not unfairly captured in terms of pleasure—a relaxed freshness, let us say, that feels wonderful.''%
\footcite[][174]{pleasure-and-desire-woolf-2009}
I'm not convinced.
This strikes me as a false interpretation of Epicurus and a dubious psychological claim.
Sometimes, people take pleasure in a lack of pain, but that seems the exception rather than the norm.

In defense of (i), Woolf explains that Epicurus argues for an incoherent position if he says that lack of pain is not itself a feeling.
Epicurus says that feeling (\textit{pathos}) determines what people pursue and what they avoid.
Since lack of physical and mental pain are our goal (\textit{telos}), and we pursue our goal, Epicurus must believe that the lack of physical and mental pain are felt as pleasures---or he is inconsistent.

Woolf makes a good---but limited---point.
We can agree that Epicurus risks incoherence, and we can agree that this possible incoherence may motivate Epicurus to insist that the lack of pain is itself pleasurable.
But \textit{we} don't have to agree.
Furthermore, we may prefer to address the incoherence in other ways, both for ourselves and for Epicurus.

Finally, Woolf relies on reasonableness and relevance to explain (ii).
Epicurus uses the cradle argument to argue for his view of happiness.
According to the cradle argument, all animals pursue pleasure and avoid pain from birth without being taught or trained.
Woolf also says that this argument would be irrelevant unless Epicurus means to say that all animals pursue a felt pleasure, but I am not sure that this is true.

Woolf himself seems to see that his argument is not conclusive.
He says that it is more plausible that all animals pursue feelings of pleasure than that they pursue a state of pleasure that is not felt.
But, of course, Epicurus may have argued for the less plausible position.
To support his interpretation, Woolf reminds us that ``it is important that the potential convert to Epicurus' philosophy be prepared to accept the point about what the young seek as a reasonable one.''
\footcite[][175]{pleasure-and-desire-woolf-2009}
Woolf believes that his interpretation gives a more reasonable position to Epicurus.
Again, however, Epicurus (and we) may disagree with Woolf about which positions are reasonable.

In terms of relevance, Woolf probably has a point, but I'm not sure that the point is conclusive.
If Epicurus argues from psychological hedonism (in the form of the cradle argument) to ethical hedonism, then we may assume that he needs a univocal meaning for hedonism.
I think that this is plausible.
However, Epicurus may be able to argue for development in the understanding of pleasure with a central core.
I would need to think more to decide about this, but there may be room for a view that is more complex but not incoherent or an equivocation.
% -]] Why is Epicurus a Hedonist? (Section VI)

% [[- Kinetic and Static Pleasures, Part 2 (Section (VII)
\section*{Kinetic and Static Pleasures, Part 2 (Section (VII)}

Woolf considers another way to connect kinetic and static pleasures in Epicurean theory.
Perhaps Epicurus believes that you can only enjoy kinetic pleasures when you are in a state of static pleasure.
First, you must be free of all pain, and then you can enjoy kinetic pleasures.
But of course the person free of all pain is already in a state of static pleasure.
Hence, says this view, Epicurus believes that kinetic pleasures rely on a pre-existing static pleasure.
Woolf attributes this view to Striker.%
\footcite{epicurean-hedonism-striker-1996}
He also refers to Cooper, though less directly.%
\footcite{pleasure-and-desire-in-epicurus-cooper-1999}

Woolf rejects this interpretation for two reasons.
First, he doubts that the texts support this interpretation.
Second, he thinks that is is false.
According to Woolf you can enjoy kinetic pleasures even if you fail to enjoy static pleasure.
He gives the example of someone enjoying a walk despite being in (some?) pain.
% -]] Kinetic and Static Pleasures, Part 2 (Section (VII)

% [[- Conclusion
\section*{Conclusion}

Woolf concludes by reiterating a few key points.

\begin{enumerate}
    \item First, the goal is static pleasure, freedom from pain, but Epicurus does not deny the reality, pleasure, or importance of kinetic pleasures.
    We would expect a life of static pleasure also to include many kinetic pleasures.
    \item Second, it remains weird that Epicurus says that pleasure cannot be increased after static pleasure is achieved.
    \item Third, tranquillity is where pleasure and happiness meet for Epicurus.
    We enjoy tranquillity as a felt pleasure, but it is also a steady, settled goal.
    As such, we can consider tranquillity as both a pleasurable state and our final goal, happiness.
    If we identify these two things, then perhaps that motivates Epicurus's denial that pleasure can be increased.
    After all, it seems intuitive that happiness cannot be increased by small things.
    Thus, if happiness \textit{is} pleasure, and happiness cannot be increased but only varied, then perhaps the same goes for pleasure.
\end{enumerate}
% -]] Conclusion

% [[- Bibliography
\newpage\
\pagestyle{references}
\printbibliography[title={Bibliography}]
% -]] Bibliography

\end{document}
% -]]

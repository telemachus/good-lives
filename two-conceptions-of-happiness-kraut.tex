% [[- LaTeX prelude
\documentclass[12pt,letterpaper]{article}

\usepackage[no-math]{fontspec}
\setmainfont{Baskerville}

% Ugh: https://tex.stackexchange.com/a/400994/29387
% This no longer seems necessary, but I’ll leave it for a bit.
% \usepackage[base]{babel}
\usepackage[nolocalmarks]{polyglossia}
\setdefaultlanguage{english}
\setotherlanguage[variant=classic]{latin}
\setotherlanguage[variant=ancient]{greek}
\newfontfamily\greekfont[Script=Greek,Scale=MatchLowercase]{GFS Neohellenic}

\usepackage{titlesec}
\titleformat*{\section}{\large\bfseries}
\titleformat*{\subsection}{\bfseries}
\titleformat*{\subsubsection}{\bfseries}

\usepackage{parskip}
\usepackage{csquotes}
\usepackage[style=windycity,citetracker=context,backend=biber]{biblatex}
\addbibresource{good-lives.bib}

\usepackage{enumitem}
\setlist{noitemsep}
\usepackage[super]{nth}

\begin{hyphenrules}{latin}
    \hyphenation{}
\end{hyphenrules}

\begin{hyphenrules}{greek}
    \hyphenation{}
\end{hyphenrules}

\usepackage{fancyhdr}
\fancypagestyle{notes}{%
    \fancyhf{}
    \renewcommand{\headrulewidth}{0pt}
    \lhead{}
    \chead{Notes on ``Two Conceptions of Happiness''}
    \rhead{}
    \lfoot{}
    \cfoot{\thepage}
    \rfoot{}
}
\fancypagestyle{references}{%
    \fancyhf{}
    \renewcommand{\headrulewidth}{0pt}
    \lhead{}
    \chead{Bibliography}
    \rhead{}
    \lfoot{}
    \cfoot{\thepage}
    \rfoot{}
}

% -]] Latex prelude

% [[- LaTeX document
\begin{document}

\raggedright

% [[- Title page
% \begin{titlepage}
% \title{Notes on Two Conceptions of Happiness (Richard Kraut)}
% \author{Peter Aronoff}
% \maketitle
% \thispagestyle{empty}
% \end{titlepage}
% -]]

\pagestyle{notes}

% [[- Introduction (Section I)
\section*{Introduction (Section I)}

Richard Kraut argues that we disagree with Aristotle over how to determine which people lead happy lives.
However, Kraut also argues that the usual way to distinguish Aristotle from us is wrong.
Most people, according to Kraut, say that happiness ``is, or at least involves, a certain state of mind'' but that \textit{eudaimonia} ``does not name a feeling or emotion.''%
\footcite[][167]{two-conceptions-of-happiness-kraut-1979}
According to Kraut, however, both happiness and \textit{eudaimonia} require ``certain attitudes towards one's life'' and that we ``measure up to certain standards.''%
\footcite[][197]{two-conceptions-of-happiness-kraut-1979}
That is, Aristotle agrees with us that \textit{eudaimonia} requires certain mental states.

In that case, where does Kraut think that we differ from Aristotle? Kraut explains as follows.

\begin{quote}
    Where we and Aristotle sharply disagree is over the standards to be used in evaluating lives.
    Roughly, he insists on an objective and stringent standard, whereas our test is more subjective and flexible.%
    \footcite[][197]{two-conceptions-of-happiness-kraut-1979}
\end{quote}

In addition, Kraut argues, contrary to a dominant scholarly view, that Aristotle does use \textit{eudaimonia} in ways parallel to our happiness.
Therefore, Kraut objects to translations of \textit{eudaimonia} as ``well-being'' or ``flourishing.''
% -]] Introduction (Section I)

% [[- Happy State of Mind (Section II)
\section*{Happy State of Mind (Section II)}

In this section, Kraut argues that Aristotle's \textit{eudaimon} is in the same positive state of mind as a contemporary happy person.
Kraut offers three arguments for this conclusion; the first relies on inference from what Aristotle says about the \textit{eudaimon}, and the other two rely on Aristotle's explicit arguments.

Here is Kraut's first argument.

\begin{enumerate}
    \item ``When we say that someone is living happily, we imply that he has certain attitudes towards his life: he is very glad to be alive; he judges that on balance his deepest desires are being satisfied and that the circumstances of his life are turning out well.''%
    \footcite[][170]{two-conceptions-of-happiness-kraut-1979}
    \item ``[The \textit{eudaimon}] loves the activities he regularly and successfully engages in.
        He thinks that exercising one's intellectual and moral capacities is the greates good available to human beings, and he knows that he possesses this good.
        Furthermore, he has all the other major goods he wants.
        His desire for such external goals as honor, wealth, and physical pleasure is moderate, and should be easy enough to satisfy in a normal life.''%
        \footcite[][170--171]{two-conceptions-of-happiness-kraut-1979}
    \item ``So, the individual who is most \textit{eudaimon} on Aristotle's theory passes our tests for happiness with flying colors.''%
        \footcite[][171]{two-conceptions-of-happiness-kraut-1979}
\end{enumerate}

Next, Kraut reminds us of something Aristotle that says explicitly.
According to Aristotle, good and fortunate people desire life most of all (\textit{NE} 1170a26--29).
Kraut says that Aristotle ``cannot mean that [\textit{the eudaimon}] will struggle to stay alive at any cost.
Rather, he must mean that such individuals are more glad to be alive than others; the kind of existence they enjoy gives them a heightened love of life.''%
\footcite[][1970]{two-conceptions-of-happiness-kraut-1979}

As part of this second argument, Kraut also points to passages where Aristotle says that people who possess \textit{eudaimonia} ``have the most reason to live'' and that ``they are the very opposite of the sort of individual who is so miserable and filled with self-hatred that he contemplates suicide.''
\footcite[][172]{two-conceptions-of-happiness-kraut-1979}
(Kraut refers to \textit{NE} 1117b9--13 and 1166b11--28 as support for his argument.)

For his final argument, Kraut turns to distinctions that Aristotle makes between humans, other animals, and plants.
(To support this argument, Kraut refers to \textit{DA} 413b1--2 and \textit{NE} 1070a16--17, where Aristotle explains that animals differ from plants because they perceive.
He also refers to \textit{EE} 1216a2--6, \textit{NE} 1095b32--35, and \textit{NE} 1176a34--35, where Aristotle says that no one would choose a lifetime of sleep since that is the life of a plant.)

\begin{enumerate}
    \item Unlike plants, says Aristotle, humans and other animals form desires for what we take to be good for us.
    \item Unlike other animals, humans form second-order desires for our first-order desires to be met.
        Non-human animals want all sorts of things, but they do not want their desires to be fulfilled.
        Thus, ``no person would choose a life in which he remains continually unaware of whether or not he possesses the good; that would be a life befitting plants, not human beings.''%
        \footcite[][172]{two-conceptions-of-happiness-kraut-1979}
    \item If a desire develops naturally and universally, we must satisfy it in order to qualify as \textit{eudaimon}.
    \item Therefore, in order to qualify as \textit{eudaimon}, we must satisfy our second-order desire to perceive that our first-order desires are met.
\end{enumerate}

Kraut supports this argument with a thought experiment.
He imagines a person who hears (falsely) that his family has been killed.
Even though this person still has every first-order good for an \textit{eudaimon} life, Kraut says that he is not \textit{eudaimon} because he lacks the self-perception that he has a good life.
In other words, the person grieves, and that grief impedes his happiness.

As a bonus, Kraut points out that his arguments work no matter what we think about the role of moral goodness in Aristotle's ideal life.
That is, Kraut makes ``no appeal\dots to Aristotle's theory that the best life is devoted to virtuous activity.''%
\footcite[][174]{two-conceptions-of-happiness-kraut-1979}
Instead, Kraut uses an argument that ``depends solely on some highly general features of human nature, and the point that a \textit{eudaimon} is without major defects.''%
\footcite[][174]{two-conceptions-of-happiness-kraut-1979}

Kraut ends this second with two clarifications.
First, he explains that he is not claiming that the word \textit{eudaimon} was only used in ordinary speech for people who felt happy or satisfied with their lives.
(Kraut seems to believe that this claim about language is true, but he doesn't require it for his argument to go through.)
Second, Kraut insists that satisfaction with one's life is a necessary but not a sufficient condition of \textit{eudaimonia} for Aristotle.
Aristotle can easily imagine cases where people are satisfied with their lives but they should not be.
Such cases, however, do not count against Kraut's larger argument unless we believe that the contemporary concept of happiness is entirely subjective.
Thus, Kraut makes a smooth transition to the next stage in his argument, where he will argue that the contemporary concept of happiness is not entirely subjective.
% -]] Happy State of Mind (Section II)

% [[- Against Extreme Subjectivism (Section III)
\section*{Against Extreme Subjectivism (Section III)}

Kraut argues that we should reject what he calls \textit{extreme subjectivism}.
Extreme subjectivism is the view that our happiness can depend on ``reality being a certain way.''%
\footcite[][178, note 21]{two-conceptions-of-happiness-kraut-1979}
According to the extreme subjectivist, ``happiness is a psychological state and nothing more; it involves, among other things, the belief that one is getting the important things one wants, as well as certain pleasant affects that normally go along with this belief.''%
\footcite[][178]{two-conceptions-of-happiness-kraut-1979}

Kraut feels that extreme subjectivism is ``a half-truth.''%
\footcite[][178]{two-conceptions-of-happiness-kraut-1979}
The true part of extreme subjectivism is that subjective feelings are an important part of happiness.
The false part of extreme subjectivism is that subjective feelings are not the whole of happiness.
Kraut believes that we care about more than what he calls ``a psychological state.''%
\footcite[][179]{two-conceptions-of-happiness-kraut-1979}

Kraut offers an example to support these claims.
He asks us to imagine someone who says that they consider a happy life one in which they have the love and admiration of friends.
This person also says that they don't want merely to think that they have this love and admiration; the person insists that they would not be satisfied by a life where they are deceived by their (so-called) friends.
On the basis of this example, Kraut argues that we would deny that the a deceived person was genuinely happy.
Thus, we should reject extreme subjectivism.

I worry that this argument relies too much on this single example.
In the given example, I agree that the person Kraut imagines is not happy by their own standard of happiness.
Their standard includes an explicit requirement that they not be deceived.
But what if we imagine a different person with no such requirement?
Or what if we set up a case where it is absolutely impossible that the deceit will ever be discovered?
How would such examples affect our intuitions?
(As an aside, I suspect that Kraut needs us a more detailed discussion of our intuitions about happiness and possibilities.)

Kraut acknowledges the pull of subjectivism, even extreme subjectivism.
He grants that we care about feeling happy.
Insofar as we want to feel happy, our understanding of happiness seems subjective.
However, Kraut denies that feeling happy is all that there is to happiness since we also want reality to match our desires and beliefs.
I agree that many people want their happiness to be real rather than illusory.
But many of the same people have conflicting intuitions about cases of deceit.
And some people say deceit is fine provided that they don't find out they were deceived.
% -]] Against Extreme Subjectivism (Section III)

% [[- A Sensible Subjectivism? (Section IV)
\section*{A Sensible Subjectivism? (Section IV)}

In this section, Kraut tries to draw a clearer line between a more limited subjectivism (a view that he favors) and objectivism about happiness.
Here is the distinction:

\begin{enumerate}
    \item A subjective view of happiness requires that people meet their own standards.
        This is key: according to the subjective view, each person can set their own standard of happiness.
    \item An objective view of happiness requires that people meet ideal standards.
        The objective view requires that people come ``reasonably close to living the best life they are capable of.''%
        \footcite[][181]{two-conceptions-of-happiness-kraut-1979}
        In the objective view, people can set whatever standards they like, but ``it is not up to you to determine where your happiness lies; it is fixed by your nature, and your job is to discover it.''%
        \footcite[][181]{two-conceptions-of-happiness-kraut-1979}
\end{enumerate}

In a nutshell, the subjectivist believes that we can (we must?) set our own goals for happiness, but the objectivist believes that we must discover what our happiness already is.

Kraut finishes this section by reminding us that objectivism and subjectivism have a great deal in common.
Both agree that a happy person believes that their life is going well and experiences satisfaction with their life.
And both agree that, in order to be truly happy, a person requires more than just good feelings about their life.
Again, the crucial difference is in what extra each requires beyond good feelings.
% -]] A Sensible Subjectivism? (Section IV)

% [[- The Difference is not Merely Verbal (Section V)
\section*{The Difference is not Merely Verbal (Section V)}

Kraut denies that the difference between objectivism and subjectivism is merely verbal.
Since both parties agree that people must actually meet a certain standard and that ``seeming to meet [their standard] is not enough,'' they are not using the word \textit{happy} in an ambiguous way.%
\footcite[][182]{two-conceptions-of-happiness-kraut-1979}
% -]] The Difference is not Merely Verbal (Section V)

% [[- Happiness, Luck, and the Future (Section VI)
\section*{Happiness, Luck, and the Future (Section VI)}

Kraut uses wishes for future well-being to better understand objectivism.
When we wish for a baby's future well-being, according to Kraut, we don't hope only that they will be satisfied with their lives.
We also hope that they will have lives worth of being satisfied with, and we hope that they will not run into the kind of bad luck that narrows or limits their futures.
(Kraut mentions blindness, enslavement, and blindness as examples of such bad luck.)

According to Kraut, subjectivists agree with all of this when they wish someone well in their (open) future.
But as people age and their choices become more limited, subjectivists are willing to settle for whatever seems worthwhile to people themselves.
Objectivists, however, remain firm.
They always believe that a truly good life requires not only subjective satisfaction but also some amount of good luck.

Kraut also uses these scenarios to argue against the idea that \textit{happiness} is ambiguous.

\begin{quote}
    Notice, by the way, how silly it would be to say that ``happiness'' has two different meanings: one when we wish children a happy life, and another when we assess the happiness of adults. Quite clearly what is happening is not a change in meaning but a change in standards. We include more in a happy life, when we wish it to the new-born, than we require of such a life, when we judge that someone has achieved it. All the more reason, then, to thin that objectivists and subjectivists mean the same by ``happiness,'' and that Aristotle's \textit{eudaimon} is properly rendered, ``leading a happy life.''%
    \footcite[][189]{two-conceptions-of-happiness-kraut-1979}
\end{quote}
% -]] Happiness, Luck, and the Future (Section VI)

% [[- The Problem with Objectivism (Section VII)
\section*{The Problem with Objectivism (Section VII)}

Kraut argues that we should reject objectivism because we lack information necessary to make the theory work.

\begin{enumerate}
    \item Objectivists owe us ``a definite idea of how to'' determine whether people are happy.%
    \footcite[][189]{two-conceptions-of-happiness-kraut-1979}
    \item Objectivists should be able to prove that people who choose the ideal life are more satisfied than those who don't.
    \item Objectivists should be able to ground their account in a a story about human nature.
    \item Objectivists should also be able to say what lives ``come \textit{reasonably} close'' to the ideal life (emphasis in the original).%
    \footcite[][190]{two-conceptions-of-happiness-kraut-1979}
    Kraut makes an interesting argument for this demand.
    He says that we cannot require perfect lives because ``there are no such lives.''%
    \footcite[][190]{two-conceptions-of-happiness-kraut-1979}
    But we need to know which among the imperfectly lived lives are close enough to consider happy and which are not.
    We need to know how to determine what is \textit{close enough} (my wording).
    Otherwise, as Kraut says, ``[i]t would be like trying to decide whether London is reasonably close to Bristol.''%
    \footcite[][190]{two-conceptions-of-happiness-kraut-1979}
\end{enumerate}

Kraut insists that he is not saying objectivism must fail but only that it has so far.
He remains open to the possibility that some objectivist theory may provide the necessary information.
But as things stand, no such theory exists, and therefore we should fall back on our subjective standards.

At the same time, Kraut is clearly not satisfied with subjectivism.
Kraut complains that subjectivism doesn't answer the question we really want answered.
People ``who are uncertain about what kind of life to lead'' want to know ``how we should lead our lives.''%
\footcite[][190]{two-conceptions-of-happiness-kraut-1979}
But subjectivism can only answer that question by telling us to ``make up our minds about what we value most.''%
\footcite[][190]{two-conceptions-of-happiness-kraut-1979}

Why does Kraut think that we shouldn't let go of subjectivism and work on pursuing a satisfying objectivism?
Kraut calls this suggestion ``quite weak.''%
\footcite[][190, note 35]{two-conceptions-of-happiness-kraut-1979}
In his mind, subjectivism is an ``adequate theory,'' and you don't dump such a theory ``because a superior view might be developed.''
\footcite[][190, note 35]{two-conceptions-of-happiness-kraut-1979}
(Note that this implies that Kraut agrees that objectivism is a superior view---or that it would be superior if it met his earlier requirements.)
% -]] The Problem with Objectivism (Section VII)

% [[- Another Objection to Aristotle (Section VIII)
\section*{Another Objection to Aristotle (Section VIII)}

Kraut complains that ``the standard by which [Aristotle] evaluates lives is too rigid.''%
\footcite[][192]{two-conceptions-of-happiness-kraut-1979}
Aristotle believes that some people, Kraut uses the example natural slaves from Aristotle, cannot reach happiness.
Such people may be able to ``achieve a low-grade form of virtue (1260a34--36),'' but that is the best they can hope for.%
\footcite[][193]{two-conceptions-of-happiness-kraut-1979}

According to Kraut, Aristotle would say the same about people with serious handicaps.
(Kraut seems to have in mind both mental and physical handicaps.)
Kraut objects that Aristotle's theory would undermine the ``self-esteem and vitality'' of such people.%
\footcite[][193]{two-conceptions-of-happiness-kraut-1979}
These people, like natural slaves according to Aristotle, cannot see themselves as conducting their lives well.
They can know that they are not leading the worst lives, but they must believe that they are leading bad lives if they know the truth.
Thus, even if Aristotle's happy life serves as an ideal for people to aim at, these people will inevitably feel miserable and defeated when they look at their prospects.

Kraut imagines a kinder objectivism as an alternative to Aristotle.
In Kraut's theory, people should guide their lives by a standard that ``reflects [their] unalterable capacities and circumstances'' rather than a single ideal for all people.
Thus, someone who cannot change their handicap can still aim at the best life possible for them, even if they cannot reach an ideal life for people in other circumstances.

Finally, Kraut returns to his starting point.
We have a substantive disagreement with Aristotle, not merely a verbal one.
If we translate \textit{eudaimonia} as \textit{flourishing}, we make a mistake and we don't solve our disagreement with Aristotle.
As Kraut says, ``[Aristotle] would accuse us, and we should accuse him, of measuring people's lives by an inappropriate standard.''%
\footcite[][197]{two-conceptions-of-happiness-kraut-1979}
% -]] Another Objection to Aristotle (Section VIII)

% [[- My Thoughts
\section*{My Thoughts}

What do I like about this article?
Kraut helpfully insists that there is a subjective element in objectivism.
A (reasonable?) objectivist will acknowledge that happy people (i) must perceive themselves to be happy and (ii) must experience some pleasurable satisfaction with their lives.


What do I dislike about this article?
Kraut also insists that there is an objective element in subjectivism, but he doesn't adequately defend this claim.
(I'll say more about this below.)
Kraut fails to look hard enough his examples, in particular the examples of slavery and people with significant mental handicaps.
He takes an enormous amount for granted and appears to assume agreement on highly controversial matters.

To expand on one criticism, I don't think that Kraut takes subjectivism nearly serious enough.
Kraut believes that there must be an objective aspect to subjectivism.
His subjectivist ``is happy because his life meets a certain standard (a subjective one).''%
\footcite[][168]{two-conceptions-of-happiness-kraut-1979}
Kraut denies that someone can be happy if they are deceived and wrongly believe that their life meets their subjective standard.
I'm not sure that Kraut is wrong, but he does nearly nothing to defend his view.
In particular, he gives only one example to support himself, and the example is clearly designed to guide readers to the conclusion that Kraut wants.
In the example, we imagine a person who sets this standard: a happy life is one where they have the love and respect of friends, but they must actually have the love and respect, they must not be deceived.
What happens, however, if we change the example to a person who says, ``My life will be happy if I think I have the love and respect of friends, and I don't care if they're lying to me, so long as I never find out about the lie''?
Kraut also fails to consider people who have wildly minimal or trivial standards for the happy life.
(E.g., the person---from an article or book that I can't remember---who wants only to raise snails.)
% -]] My Thoughts

% [[- Bibliography
\newpage\
\pagestyle{references}
\printbibliography[title={Bibliography}]
% -]] Bibliography

\end{document}
% -]]
